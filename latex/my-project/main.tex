% Document Class Declaration
\documentclass{article}

% Preamble
\usepackage[utf8]{inputenc} % For UTF-8 encoded characters
\usepackage{amsmath, amssymb, amsthm} % For maths
\usepackage{hyperref} % For links
\usepackage{graphicx, float} % For images
\graphicspath{{images/}} % Telling graphicx where our images are

% Document Customization ----------------

% Geometry
\usepackage[letterpaper, top=1in, bottom=1.0in, left=1.2in, right=1.2in, heightrounded]{geometry}

% Line Height
\renewcommand{\baselinestretch}{1.15} % Line Spacing

% Parindent & Parskip 
%\setlength{\parindent}{0pt}
%\setlength{\parskip}{0.8em}

%----------------------------------------

\title{LaTeX for Myself}
\author{Arnav Samal}
\date{May 2024}

% Document Environment
\begin{document}

\maketitle

\section{Introduction}

\subsection{What is LaTeX?}

LaTeX is a typesetting system commonly used for technical and scientific documents, particularly in academia and the publishing industry. It's based on the TeX typesetting language developed by Donald Knuth. LaTeX provides a high-level markup language that allows users to specify the structure and formatting of a document separately from its content.

It's especially popular for documents that contain complex mathematical equations, formulas, and scientific notations due to its superior handling of such elements. LaTeX documents are typically compiled into a PDF format using various LaTeX compilers such as TeX Live, MiKTeX, or Overleaf, among others.

Overall, LaTeX offers precise control over the layout and presentation of documents, making it a powerful tool for producing professional-looking papers, reports, theses, and books.
\\ 
\\ \href{https://www.overleaf.com/learn}{Check Overleaf Documentation for more}

\subsection{Basic Structure of LateX Document}
The basic structure of a LaTeX document typically consists of the following components:

\begin{description}
    \item[Document Class Declaration] This line specifies the type of document you are creating, such as an article, book, report, or presentation. For example:
    \begin{verbatim}
    \documentclass{article}
    \end{verbatim}
    
    \item[Preamble] The preamble is the section between \verb|\documentclass{}| and
    \verb|\begin{document}|. In the preamble, you can load packages, define custom commands, set document properties, and configure formatting options. For instance:
    \begin{verbatim}
    \usepackage{graphicx}
    \usepackage{amsmath}
    \title{My LaTeX Document}
    \author{John Doe}
    \date{\today}
    \end{verbatim}
    
    \item[Document Environment] The main content of your document is enclosed within the \texttt{document} environment. This environment begins with \\
    \verb|\begin{document}| and ends with \verb|\end{document}|. This is where you write the actual text of your document, including sections, paragraphs, equations, figures, tables, etc. For example:
    \begin{verbatim}
    \begin{document}
    \maketitle
    \section{Introduction}
    This is the introduction to my document.
    \section{Main Content}
    Here is the main content of the document.
    \end{document}
    \end{verbatim}
    
    \item[Title, Author, Date] These details can be specified in the preamble using commands like \verb|\title{}|, \verb|\author{}|, and \verb|\date{}|. They are usually included in the document using \verb|\maketitle| command within the document environment.
    
    \item[Sections and Text] You can organize your document into logical sections using commands like \verb|\section{}|, \verb|\subsection{}|, and \verb|\subsubsection{}|. Text content is simply written out in the document environment.
    
    \item[Mathematics] LaTeX has robust support for mathematical typesetting. You can include inline math with \$...\$ or \verb|\( ... \)| and display math with \verb|\[ ... \]| or \verb|\begin{equation} ... \end{equation}|.

    \item[Figures and Tables] Images are included using the \verb|\includegraphics{}| command  within a \texttt{figure} environment, and tables are created using the \texttt{table} environment along with commands like \verb|\begin{tabular}{...}|.

    \item[References and Citations] You can manage references and citations using packages like \texttt{biblatex} or \texttt{natbib} along with a \texttt{.bib} file containing your references.
\end{description}

\section{Mathematics}

\subsection{Writing Equations in \LaTeX}
Examples:

In 1902, Einstein created the equation: $E = mc^2$

Newton came up with this one: $\sum F=ma$
% For more fancy ones
\begin{equation}
    5+5=10
\end{equation}

\begin{equation}
    \sum_{i=1}^{\infty} \frac{1}{n^s} = \prod_p \frac{1}{1 - p^{-s}}
\end{equation}

\subsection{Links for More}

\href{https://detexify.kirelabs.org/classify.html}{Handwritten to Commands}

\section{Images}

\subsection{Including Images}
% You have to create a folder
\begin{figure}[H]
    \centering
    \includegraphics[width=5in]{brad-knight-huWlb1NP67w-unsplash}
    \caption{Chicago Skyline}
    \label{fig:enter-label}
\end{figure}

\end{document}

